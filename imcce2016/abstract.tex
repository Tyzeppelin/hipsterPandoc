\chapter{Conclusion}
\section{Français}
En conclusion, ce stage m'aura permis de me réconcillier avec l'algèbre grace à la découverte de l'algèbre computationelle qui s'est
révélé être une discipline qui mèle l'abstraction des problèmes mathématiques et la puissance des langages informatiques. J'ai pris énormément de
plaisir à travailler sur les travaux de Zassenhaus \textit{et alius} et de réfléchir à une implémentation plus ou moins efficace pour une
solution à un problème donné. La résolution algébrique d'un problème permet d'écrire une solution avec une bonne complexité,
mais comprendre la théorie n'est pas suffisant, il faut aussi savoir quelles instructions du programme prennent le plus de temps, il
faut adapter l'implémentation en fonction du fonctionnement du système de type du langage utilisé. L'inverse n'est pas bon non plus, 
si on se contente d'implémenter sans réflechir, on obtient des algorihmes "naïf". C'est ce que j'ai appris en implémentant cet algorithme
de factorisation. J'ai appris à lire des publications scientifique correctement, à travailler les algorithmes dans un langage de haut niveau
pour étudier les points critiques de ces algorithmes et de les implémenter le plus correctement possible en C++. Au final je pense avoir
fait ce qu'on attendait de moi durant ces trois mois, bien que trois mois sont probablement trop peu pour implémenter les meilleurs
algorithmes de manière optimale.


\section{Anglais}

.
