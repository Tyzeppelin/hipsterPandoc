% Table of contents formatting
\renewcommand{\contentsname}{Table of Contents}
\setcounter{tocdepth}{1}
 
% Headers and page numbering 
\usepackage{fancyhdr}
\pagestyle{plain}
 
% Fonts and typesetting
\setmainfont{TeX Gyre Pagella}
\setsansfont{Legendum}

% Set figure legends and captions to be smaller sized sans serif font
\usepackage[font={footnotesize,sf}]{caption}

\usepackage{siunitx}

% Adjust spacing between lines to 1.5
\usepackage{setspace}
\onehalfspacing
\raggedbottom

% Set margins
\usepackage[top=2.5cm, bottom=2.5cm, left=2.5cm, right=2.5cm]{geometry}

% Chapter styling
\usepackage[grey]{quotchap}
\makeatletter 
\renewcommand*{\chapnumfont}{%
  \usefont{T1}{\@defaultcnfont}{b}{n}\fontsize{70}{90}\selectfont% Default: 100/130
  \color{chaptergrey}%
}
\makeatother

% Set colour of links to black so that they don't show up when printed
\usepackage{hyperref}
\hypersetup{colorlinks=true, linkcolor=black}

% Tables
\usepackage{booktabs}
\usepackage{threeparttable}
\usepackage{array}
\newcolumntype{x}[1]{%
>{\centering\arraybackslash}m{#1}}%

% Allow for long captions and float captions on opposite page of figures 
\usepackage[rightFloats, CaptionBefore]{fltpage}

% Don't let floats cross subsections
\usepackage[section,subsection]{extraplaceins}

% Importation de pdfs
\usepackage{pdfpages}

% Juste pour avoir une "Table des matières"
\renewcommand{\contentsname}{Table des matières}

% Syntax Highlighting

\usepackage{listings}

