%% start of file `template.tex'.
%% Copyright 2006-2013 Xavier Danaux (xdanaux@gmail.com).
%
% This work may be distributed and/or modified under the
% conditions of the LaTeX Project Public License version 1.3c,
% available at http://www.latex-project.org/lppl/.


\documentclass[11pt,a4paper,sans]{moderncv}        % possible options include font size ('10pt', '11pt' and '12pt'), paper size ('a4paper', 'letterpaper', 'a5paper', 'legalpaper', 'executivepaper' and 'landscape') and font family ('sans' and 'roman')

% moderncv themes
\moderncvstyle{casual}                            % style options are 'casual' (default), 'classic', 'oldstyle' and 'banking'
\moderncvcolor{blue}                                % color options 'blue' (default), 'orange', 'green', 'red', 'purple', 'grey' and 'black'
\renewcommand{\familydefault}{\rmdefault}         % to set the default font; use '\sfdefault' for the default sans serif font, '\rmdefault' for the default roman one, or any tex font name
\nopagenumbers{}                                  % uncomment to suppress automatic page numbering for CVs longer than one page

% character encoding
\usepackage[utf8]{inputenc}                       % if you are not using xelatex ou lualatex, replace by the encoding you are using
%\usepackage{CJKutf8}                              % if you need to use CJK to typeset your resume in Chinese, Japanese or Korean

% adjust the page margins
\usepackage[scale=0.91]{geometry}
%\setlength{\hintscolumnwidth}{3cm}                % if you want to change the width of the column with the dates
%\setlength{\makecvtitlenamewidth}{10cm}           % for the 'classic' style, if you want to force the width allocated to your name and avoid line breaks. be careful though, the length is normally calculated to avoid any overlap with your personal info; use this at your own typographical risks...

% personal data
\name{François}{Boschet}
\title{Computer Science Engineer}                               % optional, remove / comment the line if not wanted
\address{Rue du brochet, 49}{1050 IXELLES}{Belgium}% optional, remove / comment the line if not wanted; the "postcode city" and and "country" arguments can be omitted or provided empty
\phone[mobile]{+33 6~24~96~40~75}                   % optional, remove / comment the line if not wanted
%\phone[fixed]{+33 2~99~60~20~79}                    % optional, remove / comment the line if not wanted
%\phone[fax]{+3~(456)~789~012}                      % optional, remove / comment the line if not wanted
%\email{francois@boschet.io}                               % optional, remove / comment the line if not wanted
\email{francois@boschet.io}
\homepage{github.com/Tyzeppelin}                         % optional, remove / comment the line if not wanted
%\extrainfo{additional information}                 % optional, remove / comment the line if not wanted
%\photo[64pt][0.4pt]{}                      % optional, remove / comment the line if not wanted; '64pt' is the height the picture must be resized to, 0.4pt is the thickness of the frame around it (put it to 0pt for no frame) and 'picture' is the name of the picture file
%\quote{Some quote}                                 % optional, remove / comment the line if not wanted

% to show numerical labels in the bibliography (default is to show no labels); only useful if you make citations in your resume
%\makeatletter
%\renewcommand*{\bibliographyitemlabel}{\@biblabel{\arabic{enumiv}}}
%\makeatother
%\renewcommand*{\bibliographyitemlabel}{[\arabic{enumiv}]}% CONSIDER REPLACING THE ABOVE BY THIS

% bibliography with mutiple entries
%\usepackage{multibib}
%\newcites{book,misc}{{Books},{Others}}
%----------------------------------------------------------------------------------
%            content
%----------------------------------------------------------------------------------
\begin{document}
%\begin{CJK*}{UTF8}{gbsn}                          % to typeset your resume in Chinese using CJK
%-----       resume       ---------------------------------------------------------
\makecvtitle

    \cventry{}{Looking for a new challenge in backend software development, available now}{}{}{}{}

\section{Education}
\cventry{2011--2017}{Master's Degree}{Institut National des Sciences Appliquées}{Rennes}{\textit{Computer Science Engineer}}{}  % arguments 3 to 6 can be left empty
\cventry{January -- June 2016}{Erasmus in Ireland}{Letterkenny Institute of Technology}{Letterkenny}{\textit{Computer Science}}{}



%\cventry{2003--2008}{DNB}{Saint Gabriel}{Pacé}{\textit{ }}{Mention Bien}

% \section{Master thesis}
% \cvitem{title}{\emph{Title}}
% \cvitem{supervisors}{Supervisors}
% \cvitem{description}{Short thesis abstract}

\section{Skills}
%\cvitemwithcomment{Français}{langue maternelle}{}
%\cvitemwithcomment{Anglais}{C1}{}
% \cvitemwithcomment{Language 3}{Skill level}{Comment}

\cvdoubleitem{Languages}{Python, C++, Java, Rust, Lisp, \ldots}{Calculus}{Numpy, Scipy, GMP}
\cvdoubleitem{OS}{Linux-based (Fedora, debian, RHEL, \ldots)}{Database}{Postgresql, MariaDB}
\cvdoubleitem{Other}{Docker, Git, Vim, Pandoc, \ldots}{Languages}{French (native), English (TOEIC 965)}


%Example --> \cventry{year--year}{Job title}{Employer}{City}{}{Description}

\section{Experience}
%\subsection{Professionnelle}

\cventry{2021}{Developper}{SpaceApplications Services}{Brussels}{}{
    \begin{itemize}
        \item Develop and maintain a web platform for planning and managing simulations of solar weather written in \textbf{Java};
        \item Develop a web platform for mission planning using \textbf{Java Quarkus};
    \end{itemize}
}

\cventry{2018 - 2020}{Developper}{Altran Belgium}{Brussels}{}{ I worked as a consulant developper on multiple project such as:
    \begin{itemize}
        \item Project EF PEB: I maintained and kept up to date a modelling software for the Région Wallone. The goal of this software was to generate energetic summary of buildings. I had to maintain and update an old \textbf{python} software that should work on recent and old windows/mac/linux;
        \item Project PLAGE: I was lead developper on a web app for Bruxelles Environnement. The app is a complex suite of forms and equations that compute the energetic consumption of a real estate and then give an objective of energy saving for each building and for the whole real estate. It was made in \textbf{Java/Spring} and \textbf{Angular};
        \item Project PEB: I helped maintaining the PEB sofware for the Belgium government. This a software that gives a buildig an energetic preformance certificate according to the european EPBD directive. It is a big \textbf{Java/Spring} application with a lot of calculus;
    \end{itemize}
%    \newline{}
}

\cventry{February - August 2017}{Study of the potential use of classical feature extraction in Deep Learning classification}{Canon Research France}{Cesson-Sévigné}{}{I've studied
    the pros and cons of using static filter/feature extraction in \textbf{Deep Learning} classification of images. \newline{}}

\cventry{Summer 2016}{Implementation of a polynomial factorization algorithm in TRIP}{Institut de Mécanique Céleste et de Calcul des Éphémérides (IMCCE)}{Paris}{}{I implemented a polynomial factorization algorithm in TRIP,
    a computer algebra system developed by and for astro-physicists, written in \textbf{C++}. \newline{}}
%\cventry{Été 2016}{Factorisation des polynômes univariés à coefficients entiers dans TRIP}{IMCCE}{Paris}{}{Travaillé sur
%TRIP, un logiciel de calcul algébrique développé à l'IMCCE. \newline{}}

\cventry{Summer 2015}{Eclipse Platform and plug-ins development}{Institut National de Recherche en Informatique et Automatique (INRIA)}{Rennes}{}{ I worked with Thomas Degueule on Melange,%Travaillé avec Thomas Degueule sur Melange, 
%une plateforme de manipulation de DSLs et leurs métamodèles. \newline{}}
an eclispe plugin that manipulate metamodels and DSLs.\newline{}}

%\cventry{Été 2014}{Développeur Java}{Fortuneo}{Brest}{}{Réalisation d'une interface utilisateur dans le cadre du projet de la Qualité de Service.\newline{}}
%\cventry{Summer 2014}{Java Developper}{Fortuneo}{Brest}{}{I developed a webpage designed to help Fortuneo visualize its website's quality of service.\newline{}}

% Detailed achievements:%
% \begin{itemize}%
% \item Achievement 1;
% \item Achievement 2, with sub-achievements:
%   \begin{itemize}%
%   \item Sub-achievement (a);
%   \item Sub-achievement (b), with sub-sub-achievements (don't do this!);
%     \begin{itemize}
%     \item Sub-sub-achievement i;
%     \item Sub-sub-achievement ii;
%     \item Sub-sub-achievement iii;
%     \end{itemize}
%   \item Sub-achievement (c);
%   \end{itemize}
% \item Achievement 3.
% \end{itemize}}

%\section{Compétences informatiques}
%\cvdoubleitem{Programmation}{C/C++, Java, Vala}{Langages de script}{Python, Ruby, Shell}
%\cvdoubleitem{Versionnement}{Git, SVN}{IDE}{Eclipse, Code::BLocks, Vim}
%\cvdoubleitem{OS}{Fedora, Debian, Windows}{Rapports}{LaTeX, Markdown, XML}
%\cvdoubleitem{Bureautique}{Microsoft Office, Libre Office}{}{}

\section{Hobbies}
%\cvitem{Musique}{}
\cvitem{Video Games}{I used to admin at \href{http://www.insalan.fr}{Insalan}, a student association organizing an e-sport event each year for over 400 players.}
\cvitem{Algorithmic}{I participate in coding contests and am solving problems on \href{https://projecteuler.net}{\color{color1}Project Euler}. My progress on \href{https://github.com/Tyzeppelin/Project-Euler}{\color{color1}Github}.}
\cvitem{Solex}{I used to repair and prepare a Solex moped for the Rock'N Solex race}


%\section{Miscellaneous}
%\cvlistitem{Driving License}
%\cvlistitem{Working on solving algorithm problems on \href{https://projecteuler.net}{Project Euler}. My progress on Github, \url{https://github.com/Tyzeppelin/Project-Euler.git}}
%\cvlistitem{Résolution des problèmes d'algorthmique sur \href{https://projecteuler.net/}{projecteuler.net}. Mon avancement sur Github, \newline{} \url{https://github.com/Tyzeppelin/Project-Euler.git}}

\end{document}

%% end of file `template.tex'.

